\documentclass{beamer}
% general info
\author{Pellegrini Stefano}
\title{\LARGE Bruteforce of a Legendre-based prng}
\date{13/12/19}
% package imports 
\usepackage{textpos} 
\usepackage{tikz,calc, svg}
\usepackage{listings}
\usepackage{amsthm, amsmath}
% edit beamer theme
\definecolor{green}{RGB}{34,83,50}
\definecolor{White}{RGB}{255,255,255}
\usetheme{boxes}
\setbeamercolor{structure}{fg=green}
\setbeamercolor{frametitle}{bg=green, fg=White}
% theorem environments
\newtheorem{thm}{Theorem}
\theoremstyle{definition}
\newtheorem{defn}[thm]{Definition}
\newtheorem{ex}[thm]{Example}
\theoremstyle{remark}
\newtheorem{nt}[thm]{Note}
% custom commands
\newcommand{\Z}{\mathbb{Z}}
\newcommand{\N}{\mathbb{N}}
\renewcommand{\L}{\mathcal{L}}
\renewcommand{\S}{\mathcal{S}}
\newcommand{\C}{\mathcal{C}}
\renewcommand{\O}{\mathcal{O}}

\begin{document}
\maketitle
\begin{frame}{Setting: PRFs}
A \textbf{Pseudo-random function (PRF)} is an \textit{efficient} emulator of a \textit{random oracle}.
\only<2->{
    \medskip
    \begin{itemize}
    \item The output of a PRF is computationally indistinguishable from random
    \begin{itemize}
    \item regardless of how the input was chosen
    \end{itemize}
    \only<3>{
        \medskip
        \item Used as building block for more complex cryptographic algorithms, such as:
        \begin{itemize}
        \item Secure encryption schemes
        \item Dynamic perfect hashing
        \item Deterministic, memoryless authentication schemes
        \end{itemize}
    }
    \end{itemize}
}
\end{frame}

\begin{frame}{Setting: MPC-friendly PRFs}
In some settings it's useful to compute PRFs in a multiparty way, so we would want:
\bigskip
\begin{itemize}
\item Few communication rounds \; $\Rightarrow$ \; \textit{low latency}
\item Long output \; $\Rightarrow$ \; \textit{high throughput}
\end{itemize}
\end{frame}

\begin{frame}{Legendre symbol PRF}
Ivan Bjerre Damg\aa rd proposed a solution at CRYPTO88:
\begin{center}\textit{``On the randomness of Legendre and Jacobi sequences''}\end{center}
\medskip
\begin{itemize}
\item Their distributions were well studied since before 1900
\item First cryptographic usage proposal
\end{itemize}
\only<2>{
    \medskip
    New candidate computationally hard problem
    \begin{itemize}
    \item Looks unrelated to classical hard tasks \\
    (RSA, DH, factorization, lattices, \ldots)
    \end{itemize}
}
\end{frame}

\begin{frame}{A bit of number-theoretic context (1/2)}
\begin{defn}[Legendre symbol]
Let $p$ be a $n$-bit prime and $a \in \Z_p^*$.
\only<3->{
    The Legendre symbol is defined in the following way:
    \only<3>{
        \[\left(\frac{a}{p}\right):=\begin{cases} 1 & a=b^2\mod{p}\; \text{ for some } b \in \Z_p^* \\
        -1 & \text{otherwise}\end{cases}\]
        For convenience $\left(\frac{0}{p}\right):=1$
    }
    \only<4>{
        \[\left(\frac{a}{p}\right)\mapsto \L_p(k):=\begin{cases} 0 & a=b^2\mod{p}\; \text{ for some } b \in \Z_p^* \\
            1 & \text{otherwise}\end{cases}\]
    }        
}
\end{defn}
\only<2>{
    \bigskip
    \begin{nt}
    Remember that
     \[\Z_p^*=\{x \in\{0,\ldots,p-1\} | \gcd(x,p)=1\}=\{1,\ldots,p-1\}\] 
     is the multiplicative group modulo $p$
    \end{nt}
}
\end{frame}

\begin{frame}{A bit of number-theoretic context (2/2)}
It follows from the Weil bound (proven by Andr\'e  Weil in 1948) that:
\bigskip
\begin{thm}
\only<1>{
    The number of occurrences of a fixed pattern of $l$ nonzero Legendre symbols among the integers $1,2,\ldots,p-1\pmod{p}$ is:
    \[\frac{p}{2^l}+\mathcal{O}(\sqrt{p}),\text{ as } p\to\infty\]
}
\only<2>{
    The distribution of fixed length substrings of Legendre symbols converges to the uniform distribution.
}
\end{thm}
\end{frame}
\begin{frame}{Legendre sequence (1/2)}
We can finally define a PRF:
\bigskip
\begin{defn}
Let $p$ be a $n$-bit prime, $k \in_R \Z_p^*$ and $l\in\N$.\\
The Legendre sequence is:
\[\S(k,l) := \L_p(k), \L_p(k+1), \ldots,\L_p(k+l-1)\in \{0,1\}^l\]
\end{defn}
\end{frame}
\begin{frame}{Legendre sequence (2/2)}
\begin{itemize}
\item Damg\aa rd conjectured additional pseudo-randomness properties of this sequence in his proposal to CRYPTO88:
\medskip
\item Renewed interest in this topic arose with a 2016 paper by L. Grassi et al.: \textit{``MPC-Friendly Symmetric Key Primitives''}
\medskip
\item Attention of the \textit{ethereum} community for \textbf{proof-of-custody}.
\begin{itemize}
\item Ethereum 2.0: based on proof of stake
\item Prizes for effective attacks
\end{itemize}
\end{itemize}
\end{frame}
\begin{frame}{Ethereum bounties}
The Ethereum community offers monetary rewards for:
\begin{itemize}
\only<1>{
    \item Subexponential key recovery algorithm or security proof;
    \item Improvement on Khovratovich key recovery attack;
    \item One research paper regarding Legendre PRF;
}
\item Breaking computational hardness assumptions.
\end{itemize}
\only<2>{
    \bigskip
    \includegraphics[width=\textwidth, keepaspectratio]{figs/bounty}
}
\end{frame}
\begin{frame}{The problem}
Given a fixed-size Legendre PRF output 
\[\S(k,l) := \L_p(k), \L_p(k+1), \ldots,\L_p(k+l-1)\]
find the key $k$.
\end{frame}
\begin{frame}{My approach (1/2)}
An optimized (as much as possible) bruteforce attack:
\bigskip
\begin{itemize}
\item Minimize the number of Legendre symbol calculations
\item No asymptotic complexity improvement, just constant
\item Optimization of the deriving overload
\end{itemize}
\end{frame}
\begin{frame}{My approach (2/2)}
An optimized (as much as possible) bruteforce attack:
\bigskip
\begin{itemize}
\item Considering candidate key $c$, for a $l$-long hint:
\begin{itemize}
    \item Calculate $\L_p(c+l)$;
    \item Remove the incompatible candidates from $(c, \ldots, c+l)$;
    \item Calculate symbols backwards till $c$ also gets excluded:
    \begin{itemize}
        \item While keeping track of the set of candidates $\C$;
    \end{itemize}
    \item $c\leftarrow \min \C$
\end{itemize}
\end{itemize}
\end{frame}

\begin{frame}{Obtained results: best}
\begin{itemize}
\item keyspace of size \texttt{32-bits}
\item $p$ of size \texttt{80-bits}
\item hint of size \texttt{1000-bits}
\end{itemize} 
\end{frame}
\begin{frame}{Obtained results: keyspace size}
\includesvg[width=\textwidth]{figs/inc-keyspace-time}
\end{frame}
\begin{frame}{Obtained results: hint length}
\includesvg[width=\textwidth]{figs/inc-hint_bitlength-time}
\end{frame}
\begin{frame}{Obtained results: prime size in bits}
\includesvg[width=\textwidth]{figs/inc-log_2(p)-time}
\end{frame}
\begin{frame}{Further improvement: algorithmic}
With prime $p$, uniformly distributed key, with $M$ PRF queries needed:
\bigskip
\begin{itemize}
\item Khovratovich key recovery attack runs in $\O(p\frac{\log(M)}{M})$
\medskip
\item Ward Beullens claimed the bounty with $\O(p\frac{\log^2(p)}{M}), M<p^\frac14$
\end{itemize}
\end{frame}


\end{document}
